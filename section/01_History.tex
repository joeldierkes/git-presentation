\section{Historie}

\begin{frame}{Probleme}
  \textbf{Was für Hauptprobleme gibt es in der Softwareentwicklung?}
  \begin{itemize}
    \pause
    \item Viele Entwickler, die \textbf{gleichzeitig} arbeiten
    \pause
    \item Durch Veränderungen am Code können \textbf{Fehler} entstehen
  \end{itemize}

  \note[item]{\textbf{Entwickler}: Zwei Entwickler arbeiten an der gleichen Datei/Zeile, welche soll genommen werden?}
  \note[item]{\textbf{Fehler}: Wann ist dieser Fehler entstanden, wann hat das System noch funktioniert?}
  \note[item]{$\rightarrow$ Offensichtlich muss Code nachvollziehbar versioniert werden, mit Datum, Uhrzeit, Entwickler, ...}
\end{frame}

\begin{frame}{Versionsverwaltung}
  \begin{Definition}
    Eine \textbf{Versionsverwaltung} ist ein System, das zur Erfassung von Änderungen an Dokumenten oder Dateien verwendet wird. \cite{WVW}
  \end{Definition}

  \begin{itemize}
    \pause
    \item Zentrale Systeme:
      \begin{itemize}
        \item Media Wiki
        \item SCCS
        \item ...
      \end{itemize}
    \pause
    \item Verteilte Systeme:
      \begin{itemize}
        \item \textbf{Git}\cite{WGIT}
        \item BitKeeper
        \item Mercurial
        \item ...
      \end{itemize}
  \end{itemize}
  
  \note[item]{\textbf{Zentrale} Systeme haben ein Repository, auf dem Änderungen gespeichert werden}
  \note[item]{Bei \textbf{verteilten} Systemen besitzt jeder Nutzer eine Kopie des Repositories und diese werden dann miteinander (meist einem zentralen) abgeglichen}
\end{frame}

\begin{frame}{Git}
  \begin{itemize}
    \item Autor: \textbf{Linus Torvalds}
    \pause
    \item Veröffentlicht am: \textbf{7. April 2005}
    \pause
    \item \textbf{Open Source}: \url{https://github.com/git/git}
    \pause
    \item Sollte \textbf{schnell}, \textbf{performant} und \textbf{einfach zu bedienen} sein
  \end{itemize}

  \note[item]{Für \textbf{Linux} entwickelt, da kein System den Standards von Linus genügte}
  \note[item]{Very \textbf{first Commit}: \glqq Initial revision of "git", the information manager from hell\grqq}
\end{frame}
